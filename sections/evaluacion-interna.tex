\subsection{Evaluación Interna -- Perfil de Capacidad Interna (PCI)}
\label{sec:evaluacion_interna}

\subsubsection{Marco Metodológico PCI}

El Perfil de Capacidad Interna (PCI) es una herramienta de diagnóstico estratégico que permite evaluar sistemáticamente las fortalezas y debilidades organizacionales a través del análisis de capacidades, recursos y competencias internas. Este análisis facilita la identificación de ventajas competitivas sostenibles y áreas de mejora críticas para el desempeño organizacional \autocite{barney1991}.

La metodología PCI se fundamenta en la teoría de recursos y capacidades, que sostiene que las ventajas competitivas duraderas provienen de recursos internos valiosos, raros, difíciles de imitar y debidamente organizados. Para MercadoLibre, este análisis es crucial dado su modelo de negocio que integra tecnología, servicios financieros y operaciones logísticas a escala latinoamericana \autocite{teece2007}.

\subsubsection{Unidades de Análisis PCI}

Para la evaluación interna de MercadoLibre se han identificado ocho unidades de análisis que representan las capacidades organizacionales críticas:

\begin{enumerate}
\item \textbf{Recursos Humanos}: Gestión del talento, capacitación continua, tasas de retención, clima organizacional y cultura corporativa.

\item \textbf{Tecnología y Sistemas}: Infraestructura tecnológica, automatización de procesos, seguridad informática, escalabilidad de plataformas y capacidades de integración sistémica.

\item \textbf{Capacidades Financieras}: Solidez financiera, liquidez operativa, rentabilidad, acceso a mercados de capital y gestión de flujos de efectivo.

\item \textbf{Procesos Operativos}: Eficiencia operacional, calidad del servicio, gestión de riesgos, optimización logística y tiempos de respuesta.

\item \textbf{Marketing y Ventas}: Participación de mercado, efectividad de canales digitales, satisfacción del cliente y posicionamiento de marca.

\item \textbf{Gestión de Proveedores}: Calidad de proveedores, relaciones estratégicas, gestión de costos y diversificación de la cadena de suministro.

\item \textbf{Innovación y Desarrollo}: Capacidad de innovación, inversión en I+D, desarrollo de nuevos productos y servicios, y velocidad de implementación.

\item \textbf{Relación con Clientes}: Programas de fidelización, calidad del soporte técnico, gestión de feedback y experiencia del usuario.
\end{enumerate}

\subsubsection{Variables por Unidad de Análisis}

\subsubsection{Recursos Humanos}

La gestión de recursos humanos constituye un factor crítico para MercadoLibre, especialmente considerando la alta demanda de talento tecnológico especializado en el mercado latinoamericano.

\begin{itemize}
\item \textbf{Programas de Capacitación Técnica y Desarrollo Profesional}: Evalúa la robustez y efectividad de los programas internos de formación continua, certificaciones técnicas y desarrollo de competencias digitales. Incluye inversión en educación formal, bootcamps internos y programas de mentoring que permiten mantener al equipo actualizado con las últimas tecnologías y metodologías.

\item \textbf{Tasas de Retención de Talento Clave}: Analiza la capacidad organizacional para retener profesionales críticos, especialmente en áreas de ingeniería, ciencia de datos, seguridad informática y desarrollo de productos. Considera factores como satisfacción laboral, compensación competitiva y oportunidades de crecimiento profesional.

\item \textbf{Clima Organizacional y Satisfacción Laboral}: Mide el ambiente de trabajo, cultura corporativa, nivel de engagement de los empleados y percepción de liderazgo. Incluye métricas como índices de satisfacción laboral, encuestas de clima organizacional y evaluaciones de cultura corporativa.

\item \textbf{Cultura de Innovación y Colaboración}: Evalúa el grado en que la organización fomenta la experimentación, el aprendizaje de fallos, la colaboración cross-funcional y la generación de ideas innovadoras. Considera programas internos de innovación, hackathons y espacios para la creatividad empresarial.
\end{itemize}

\subsubsection{Tecnología y Sistemas}

Como empresa de base tecnológica, las capacidades en tecnología y sistemas constituyen el núcleo de las ventajas competitivas de MercadoLibre.

\begin{itemize}
\item \textbf{Nivel de Automatización de Procesos Críticos}: Analiza el grado de automatización en procesos core como gestión de inventario, procesamiento de pagos, detección de fraude, recomendaciones de productos y gestión logística. Evalúa la eficiencia operacional y capacidad de escalabilidad sin incremento proporcional de costos.

\item \textbf{Sistemas de Seguridad Informática y Prevención de Fraude}: Examina la robustez de la infraestructura de ciberseguridad, sistemas de detección y prevención de fraude, protección de datos personales y cumplimiento de estándares internacionales como PCI DSS para manejo de información financiera.

\item \textbf{Escalabilidad de Infraestructura Tecnológica}: Evalúa la capacidad de la arquitectura tecnológica para soportar crecimiento exponencial en transacciones, usuarios concurrentes y volumen de datos sin degradación de performance. Incluye arquitectura de microservicios, capacidad de procesamiento distribuido y elasticidad de la infraestructura cloud.

\item \textbf{Capacidades de Integración con Sistemas Externos}: Analiza la facilidad y eficiencia para integrarse con sistemas de terceros como bancos, transportadoras, proveedores de pagos, sistemas ERP de vendedores y APIs de servicios complementarios.
\end{itemize}

\subsubsection{Capacidades Financieras}

Las capacidades financieras determinan la solidez económica y la capacidad de inversión sostenida en crecimiento e innovación.

\begin{itemize}
\item \textbf{Índices de Liquidez y Solvencia}: Evalúa la salud financiera a través de métricas como liquidez corriente, prueba ácida, cobertura de deuda y estructura de capital. Analiza la capacidad para cumplir obligaciones de corto y largo plazo mientras mantiene flexibilidad financiera para inversiones estratégicas.

\item \textbf{Márgenes de Rentabilidad (EBITDA, ROE, ROA)}: Examina la eficiencia en la generación de valor a través de márgenes operativos, EBITDA, retorno sobre patrimonio y retorno sobre activos. Considera la evolución de la rentabilidad y comparación con benchmarks del sector.

\item \textbf{Acceso a Financiamiento y Mercados de Capital}: Analiza la capacidad para acceder a fuentes de financiamiento diversificadas, incluyendo mercados de capital, financiamiento bancario, bonos corporativos y capacidad de generar flujo de caja libre para autofinanciar crecimiento.

\item \textbf{Gestión de Riesgos Financieros y Cambiarios}: Evalúa los sistemas de gestión de riesgos financieros, incluyendo exposición cambiaria por operaciones en múltiples países, riesgo crediticio en servicios financieros y gestión de liquidez en diferentes monedas.
\end{itemize}

\subsubsection{Procesos Operativos}

Los procesos operativos determinan la eficiencia en la ejecución del modelo de negocio y la calidad de la experiencia del usuario.

\begin{itemize}
\item \textbf{Eficiencia en Procesamiento de Transacciones}: Mide la velocidad, precisión y costo de procesamiento de transacciones comerciales y financieras. Incluye métricas como tiempo de respuesta, tasa de éxito de transacciones y costo por transacción procesada.

\item \textbf{Calidad del Servicio Logístico (Tiempos de Entrega)}: Evalúa el desempeño de la red logística de Mercado Envíos, incluyendo tiempos de entrega, tasa de entregas exitosas, gestión de devoluciones y satisfacción del cliente con el servicio logístico.

\item \textbf{Sistemas de Gestión de Riesgos Operacionales}: Analiza los mecanismos para identificar, evaluar y mitigar riesgos operativos como fallas de sistemas, interrupciones de servicio, errores de proceso y contingencias operacionales.

\item \textbf{Optimización de Costos Operativos}: Examina la eficiencia en la gestión de costos operativos, incluyendo optimización de procesos, automatización de tareas repetitivas y gestión eficiente de recursos para mantener márgenes competitivos.
\end{itemize}

\subsubsection{Marketing y Ventas}

Las capacidades de marketing y ventas determinan la efectividad en la adquisición, retención y monetización de usuarios en el ecosistema.

\begin{itemize}
\item \textbf{Participación de Mercado en Segmentos Clave}: Evalúa la posición competitiva en diferentes categorías de productos, segmentos demográficos y mercados geográficos. Analiza la evolución de market share y penetración en segmentos estratégicos.

\item \textbf{Net Promoter Score (NPS) y Satisfacción del Cliente}: Mide la lealtad de los usuarios y su disposición a recomendar la plataforma. Incluye métricas de satisfacción del cliente, tasas de churn y lifetime value de usuarios.

\item \textbf{Efectividad de Marketing Digital y Adquisición de Usuarios}: Analiza la eficiencia de los canales de marketing digital, costo de adquisición de clientes (CAC), retorno sobre inversión en marketing (ROAS) y efectividad de campañas de performance marketing.

\item \textbf{Fortaleza de Marca y Reconocimiento Regional}: Evalúa el valor de la marca MercadoLibre, reconocimiento espontáneo y asistido, asociaciones de marca y posicionamiento versus competidores en los mercados objetivo.
\end{itemize}

\subsubsection{Gestión de Proveedores}

La gestión eficiente de proveedores es crucial para mantener la calidad del ecosistema y optimizar costos operativos.

\begin{itemize}
\item \textbf{Evaluación de Calidad de Proveedores Estratégicos}: Analiza los mecanismos de evaluación y monitoreo de proveedores críticos como servicios de cloud computing, proveedores logísticos, servicios de pago y socios tecnológicos clave.

\item \textbf{Relaciones de Largo Plazo con Vendedores}: Evalúa la gestión de la relación con vendedores en la plataforma, incluyendo programas de desarrollo de vendedores, herramientas de gestión y servicios de valor agregado que fortalecen el ecosistema.

\item \textbf{Gestión de Costos y Eficiencia de Negociación}: Examina la capacidad para negociar términos favorables con proveedores, optimizar costos de servicios y mantener relaciones comerciales que generen valor mutuo.

\item \textbf{Diversificación de la Base de Proveedores}: Analiza el nivel de dependencia de proveedores específicos y las estrategias de diversificación para mitigar riesgos de concentración y asegurar continuidad operacional.
\end{itemize}

\subsubsection{Innovación y Desarrollo}

Las capacidades de innovación determinan la capacidad de MercadoLibre para mantener liderazgo tecnológico y desarrollar nuevos productos y servicios.

\begin{itemize}
\item \textbf{Capacidad de Desarrollo de Nuevos Productos}: Evalúa la efectividad del proceso de innovación, desde la ideación hasta el lanzamiento de nuevos productos y servicios. Incluye metodologías de desarrollo ágil, testing de mercado y capacidad de iteración rápida.

\item \textbf{Inversión en Investigación y Desarrollo}: Analiza el nivel de inversión en I+D como porcentaje de ingresos, calidad de los proyectos de investigación y alineación de la inversión en I+D con objetivos estratégicos de largo plazo.

\item \textbf{Proyectos de Innovación Tecnológica en Curso}: Examina la cartera de proyectos de innovación, incluyendo inteligencia artificial, machine learning, blockchain, y otras tecnologías emergentes que pueden generar ventajas competitivas futuras.

\item \textbf{Velocidad de Time-to-Market para Nuevas Funcionalidades}: Mide la agilidad organizacional para llevar nuevas funcionalidades desde la concepción hasta el mercado, incluyendo procesos de desarrollo, testing y deployment.
\end{itemize}

\subsubsection{Relación con Clientes}

La gestión de la relación con clientes determina la capacidad de generar valor de largo plazo y construir lealtad en el ecosistema.

\begin{itemize}
\item \textbf{Programas de Fidelización y Retención}: Evalúa la efectividad de programas como Mercado Puntos, beneficios para usuarios frecuentes y estrategias de retención que aumentan el lifetime value de los clientes.

\item \textbf{Calidad del Soporte Técnico y Atención al Cliente}: Analiza la eficiencia y calidad del servicio de atención al cliente, incluyendo tiempos de respuesta, tasa de resolución en primer contacto y satisfacción con el soporte recibido.

\item \textbf{Sistemas de Gestión de Feedback y Mejora Continua}: Examina los mecanismos para capturar, analizar y actuar sobre el feedback de usuarios, incluyendo sistemas de reputación, surveys y análisis de sentimiento.

\item \textbf{Personalización de la Experiencia del Usuario}: Evalúa las capacidades de personalización de la plataforma, incluyendo recomendaciones de productos, contenido personalizado y experiencias adaptadas a los comportamientos y preferencias individuales de los usuarios.
\end{itemize}

\subsubsection{Metodología de Evaluación PCI}

La evaluación se realiza mediante la asignación de valores de importancia (1--5) e impacto interno ($-3$ a $+3$) para cada variable:

\begin{itemize}
\item \textbf{Escala de Importancia}: 1 = Muy baja, 2 = Baja, 3 = Media, 4 = Alta, 5 = Muy alta
\item \textbf{Escala de Impacto}: $-3$ = Debilidad crítica, $-2$ = Debilidad significativa, $-1$ = Debilidad menor, 0 = Factor neutro, $+1$ = Fortaleza menor, $+2$ = Fortaleza significativa, $+3$ = Fortaleza distintiva
\end{itemize}

La evaluación ponderada se calcula multiplicando la importancia por el impacto para cada variable analizada.

\subsubsection{Matriz PCI -- Perfil de Capacidad Interna}

\footnotesize
\begin{longtable}{|p{2.2cm}|p{2.8cm}|c|c|c|c|c|c|c|c|c|c|}
\hline
\multirow{2}{*}{\textbf{Unidad}} & \multirow{2}{*}{\textbf{Variables}} & \multirow{2}{*}{\textbf{Imp.}} & \multirow{2}{*}{\textbf{Impacto}} & \multirow{2}{*}{\textbf{Pond.}} & \multicolumn{7}{c|}{\textbf{Escala de Evaluación}} \\
\cline{6-12}
& & & & & \textbf{-15} & \textbf{-10} & \textbf{-5} & \textbf{0} & \textbf{5} & \textbf{10} & \textbf{15} \\
\hline
\endfirsthead

\multicolumn{12}{c}%
{{\bfseries \tablename\ \thetable{} -- continuación de la página anterior}} \\
\hline
\multirow{2}{*}{\textbf{Unidad}} & \multirow{2}{*}{\textbf{Variables}} & \multirow{2}{*}{\textbf{Imp.}} & \multirow{2}{*}{\textbf{Impacto}} & \multirow{2}{*}{\textbf{Pond.}} & \multicolumn{7}{c|}{\textbf{Escala de Evaluación}} \\
\cline{6-12}
& & & & & \textbf{-15} & \textbf{-10} & \textbf{-5} & \textbf{0} & \textbf{5} & \textbf{10} & \textbf{15} \\
\hline
\endhead

\hline \multicolumn{12}{|r|}{{Continúa en la siguiente página}} \\ \hline
\endfoot

\hline
\endlastfoot
\multirow{4}{*}{\makecell{Recursos\\Humanos}} 
& Capacitación y Desarrollo & 5 & 3 & 15 & \multicolumn{1}{c|}{} & \multicolumn{1}{c|}{} & \multicolumn{1}{c|}{} & \multicolumn{1}{c|}{} & \multicolumn{1}{c|}{} & \multicolumn{1}{c|}{} & \textbullet \\
\cline{2-12}
& Retención de Talento & 4 & 2 & 8 & \multicolumn{1}{c|}{} & \multicolumn{1}{c|}{} & \multicolumn{1}{c|}{} & \multicolumn{1}{c|}{} & \multicolumn{1}{c|}{} & \multicolumn{1}{c|}{\textbullet} & \\
\cline{2-12}
& Clima Organizacional & 4 & 3 & 12 & \multicolumn{1}{c|}{} & \multicolumn{1}{c|}{} & \multicolumn{1}{c|}{} & \multicolumn{1}{c|}{} & \multicolumn{1}{c|}{} & \multicolumn{1}{c|}{} & \textbullet \\
\cline{2-12}
& Cultura de Innovación & 5 & 3 & 15 & \multicolumn{1}{c|}{} & \multicolumn{1}{c|}{} & \multicolumn{1}{c|}{} & \multicolumn{1}{c|}{} & \multicolumn{1}{c|}{} & \multicolumn{1}{c|}{} & \textbullet \\
\hline
\multirow{4}{*}{\makecell{Tecnología\\y Sistemas}} 
& Automatización & 5 & 3 & 15 & \multicolumn{1}{c|}{} & \multicolumn{1}{c|}{} & \multicolumn{1}{c|}{} & \multicolumn{1}{c|}{} & \multicolumn{1}{c|}{} & \multicolumn{1}{c|}{} & \textbullet \\
\cline{2-12}
& Seguridad Informática & 5 & 3 & 15 & \multicolumn{1}{c|}{} & \multicolumn{1}{c|}{} & \multicolumn{1}{c|}{} & \multicolumn{1}{c|}{} & \multicolumn{1}{c|}{} & \multicolumn{1}{c|}{} & \textbullet \\
\cline{2-12}
& Escalabilidad & 5 & 3 & 15 & \multicolumn{1}{c|}{} & \multicolumn{1}{c|}{} & \multicolumn{1}{c|}{} & \multicolumn{1}{c|}{} & \multicolumn{1}{c|}{} & \multicolumn{1}{c|}{} & \textbullet \\
\cline{2-12}
& Integración Sistémica & 4 & 2 & 8 & \multicolumn{1}{c|}{} & \multicolumn{1}{c|}{} & \multicolumn{1}{c|}{} & \multicolumn{1}{c|}{} & \multicolumn{1}{c|}{} & \multicolumn{1}{c|}{\textbullet} & \\
\hline
\multirow{4}{*}{\makecell{Capacidades\\Financieras}} 
& Liquidez y Solvencia & 5 & 2 & 10 & \multicolumn{1}{c|}{} & \multicolumn{1}{c|}{} & \multicolumn{1}{c|}{} & \multicolumn{1}{c|}{} & \multicolumn{1}{c|}{} & \multicolumn{1}{c|}{\textbullet} & \\
\cline{2-12}
& Rentabilidad & 4 & 1 & 4 & \multicolumn{1}{c|}{} & \multicolumn{1}{c|}{} & \multicolumn{1}{c|}{} & \multicolumn{1}{c|}{} & \multicolumn{1}{c|}{\textbullet} & \multicolumn{1}{c|}{} & \\
\cline{2-12}
& Acceso a Financiamiento & 5 & 3 & 15 & \multicolumn{1}{c|}{} & \multicolumn{1}{c|}{} & \multicolumn{1}{c|}{} & \multicolumn{1}{c|}{} & \multicolumn{1}{c|}{} & \multicolumn{1}{c|}{} & \textbullet \\
\cline{2-12}
& Gestión de Riesgos & 4 & -1 & -4 & \multicolumn{1}{c|}{} & \multicolumn{1}{c|}{} & \multicolumn{1}{c|}{\textbullet} & \multicolumn{1}{c|}{} & \multicolumn{1}{c|}{} & \multicolumn{1}{c|}{} & \\
\hline
\multirow{4}{*}{\makecell{Procesos\\Operativos}} 
& Eficiencia Operacional & 5 & 2 & 10 & \multicolumn{1}{c|}{} & \multicolumn{1}{c|}{} & \multicolumn{1}{c|}{} & \multicolumn{1}{c|}{} & \multicolumn{1}{c|}{} & \multicolumn{1}{c|}{\textbullet} & \\
\cline{2-12}
& Calidad del Servicio & 5 & 3 & 15 & \multicolumn{1}{c|}{} & \multicolumn{1}{c|}{} & \multicolumn{1}{c|}{} & \multicolumn{1}{c|}{} & \multicolumn{1}{c|}{} & \multicolumn{1}{c|}{} & \textbullet \\
\cline{2-12}
& Gestión de Riesgos Op. & 4 & 1 & 4 & \multicolumn{1}{c|}{} & \multicolumn{1}{c|}{} & \multicolumn{1}{c|}{} & \multicolumn{1}{c|}{} & \multicolumn{1}{c|}{\textbullet} & \multicolumn{1}{c|}{} & \\
\cline{2-12}
& Optimización Costos & 3 & 0 & 0 & \multicolumn{1}{c|}{} & \multicolumn{1}{c|}{} & \multicolumn{1}{c|}{} & \multicolumn{1}{c|}{\textbullet} & \multicolumn{1}{c|}{} & \multicolumn{1}{c|}{} & \\
\hline
\multirow{4}{*}{\makecell{Marketing\\y Ventas}} 
& Participación de Mercado & 5 & 3 & 15 & \multicolumn{1}{c|}{} & \multicolumn{1}{c|}{} & \multicolumn{1}{c|}{} & \multicolumn{1}{c|}{} & \multicolumn{1}{c|}{} & \multicolumn{1}{c|}{} & \textbullet \\
\cline{2-12}
& Satisfacción del Cliente & 5 & 2 & 10 & \multicolumn{1}{c|}{} & \multicolumn{1}{c|}{} & \multicolumn{1}{c|}{} & \multicolumn{1}{c|}{} & \multicolumn{1}{c|}{} & \multicolumn{1}{c|}{\textbullet} & \\
\cline{2-12}
& Efectividad Digital & 4 & 2 & 8 & \multicolumn{1}{c|}{} & \multicolumn{1}{c|}{} & \multicolumn{1}{c|}{} & \multicolumn{1}{c|}{} & \multicolumn{1}{c|}{} & \multicolumn{1}{c|}{\textbullet} & \\
\cline{2-12}
& Fortaleza de Marca & 5 & 3 & 15 & \multicolumn{1}{c|}{} & \multicolumn{1}{c|}{} & \multicolumn{1}{c|}{} & \multicolumn{1}{c|}{} & \multicolumn{1}{c|}{} & \multicolumn{1}{c|}{} & \textbullet \\
\hline
\multirow{3}{*}{\makecell{Gestión de\\Proveedores}} 
& Calidad Proveedores & 4 & 2 & 8 & \multicolumn{1}{c|}{} & \multicolumn{1}{c|}{} & \multicolumn{1}{c|}{} & \multicolumn{1}{c|}{} & \multicolumn{1}{c|}{} & \multicolumn{1}{c|}{\textbullet} & \\
\cline{2-12}
& Relaciones Estratégicas & 3 & 1 & 3 & \multicolumn{1}{c|}{} & \multicolumn{1}{c|}{} & \multicolumn{1}{c|}{} & \multicolumn{1}{c|}{} & \multicolumn{1}{c|}{\textbullet} & \multicolumn{1}{c|}{} & \\
\cline{2-12}
& Gestión de Costos & 4 & 1 & 4 & \multicolumn{1}{c|}{} & \multicolumn{1}{c|}{} & \multicolumn{1}{c|}{} & \multicolumn{1}{c|}{} & \multicolumn{1}{c|}{\textbullet} & \multicolumn{1}{c|}{} & \\
\hline
\multirow{4}{*}{\makecell{Innovación\\y Desarrollo}} 
& Capacidad de Innovación & 5 & 3 & 15 & \multicolumn{1}{c|}{} & \multicolumn{1}{c|}{} & \multicolumn{1}{c|}{} & \multicolumn{1}{c|}{} & \multicolumn{1}{c|}{} & \multicolumn{1}{c|}{} & \textbullet \\
\cline{2-12}
& Inversión en I+D & 5 & 3 & 15 & \multicolumn{1}{c|}{} & \multicolumn{1}{c|}{} & \multicolumn{1}{c|}{} & \multicolumn{1}{c|}{} & \multicolumn{1}{c|}{} & \multicolumn{1}{c|}{} & \textbullet \\
\cline{2-12}
& Proyectos Innovación & 4 & 3 & 12 & \multicolumn{1}{c|}{} & \multicolumn{1}{c|}{} & \multicolumn{1}{c|}{} & \multicolumn{1}{c|}{} & \multicolumn{1}{c|}{} & \multicolumn{1}{c|}{} & \textbullet \\
\cline{2-12}
& Velocidad Time-to-Market & 4 & 2 & 8 & \multicolumn{1}{c|}{} & \multicolumn{1}{c|}{} & \multicolumn{1}{c|}{} & \multicolumn{1}{c|}{} & \multicolumn{1}{c|}{} & \multicolumn{1}{c|}{\textbullet} & \\
\hline
\multirow{4}{*}{\makecell{Relación\\con Clientes}} 
& Programas Fidelización & 4 & 2 & 8 & \multicolumn{1}{c|}{} & \multicolumn{1}{c|}{} & \multicolumn{1}{c|}{} & \multicolumn{1}{c|}{} & \multicolumn{1}{c|}{} & \multicolumn{1}{c|}{\textbullet} & \\
\cline{2-12}
& Calidad de Soporte & 5 & 2 & 10 & \multicolumn{1}{c|}{} & \multicolumn{1}{c|}{} & \multicolumn{1}{c|}{} & \multicolumn{1}{c|}{} & \multicolumn{1}{c|}{} & \multicolumn{1}{c|}{\textbullet} & \\
\cline{2-12}
& Gestión de Feedback & 4 & 3 & 12 & \multicolumn{1}{c|}{} & \multicolumn{1}{c|}{} & \multicolumn{1}{c|}{} & \multicolumn{1}{c|}{} & \multicolumn{1}{c|}{} & \multicolumn{1}{c|}{} & \textbullet \\
\cline{2-12}
& Personalización UX & 5 & 3 & 15 & \multicolumn{1}{c|}{} & \multicolumn{1}{c|}{} & \multicolumn{1}{c|}{} & \multicolumn{1}{c|}{} & \multicolumn{1}{c|}{} & \multicolumn{1}{c|}{} & \textbullet \\
\hline
\caption{Matriz PCI -- Evaluación de Capacidades Internas de MercadoLibre}
\label{tab:matriz_pci}
\end{longtable}
\normalsize

La Tabla \ref{tab:matriz_pci} presenta la evaluación detallada de todas las variables consideradas en el análisis PCI, permitiendo identificar las fortalezas y debilidades organizacionales de MercadoLibre.

\subsubsection{Resultados de la Evaluación PCI}

\begin{table}[H]
\centering
\begin{tabular}{|l|c|c|}
\hline
\textbf{Unidad de Análisis} & \textbf{Ponderación Total} & \textbf{Clasificación} \\
\hline
Innovación y Desarrollo & +50 & Fortaleza Distintiva \\
\hline
Tecnología y Sistemas & +53 & Fortaleza Distintiva \\
\hline
Marketing y Ventas & +48 & Fortaleza Distintiva \\
\hline
Recursos Humanos & +50 & Fortaleza Distintiva \\
\hline
Relación con Clientes & +45 & Fortaleza Significativa \\
\hline
Procesos Operativos & +29 & Fortaleza Significativa \\
\hline
Capacidades Financieras & +25 & Fortaleza Significativa \\
\hline
Gestión de Proveedores & +15 & Fortaleza Minor \\
\hline
\end{tabular}
\caption{Evaluación Ponderada por Unidad de Análisis -- PCI}
\label{tab:resultados_pci}
\end{table}

La Tabla \ref{tab:resultados_pci} consolida los resultados ponderados por cada unidad de análisis, mostrando que MercadoLibre posee fortalezas distintivas en las áreas críticas para su modelo de negocio.

\subsubsection{Análisis e Interpretación de Resultados PCI}

\paragraph{Principales Fortalezas Identificadas}

\begin{enumerate}
\item \textbf{Tecnología y Sistemas (Ponderada: +53)}: La infraestructura tecnológica de MercadoLibre constituye su ventaja competitiva más robusta, con altos niveles de automatización, seguridad informática avanzada y escalabilidad demostrada para soportar millones de transacciones diarias.

\item \textbf{Innovación y Desarrollo (Ponderada: +50)}: La capacidad continua de innovación y la inversión sostenida en I+D permiten a MercadoLibre mantener liderazgo en desarrollo de nuevos productos y servicios financieros.

\item \textbf{Recursos Humanos (Ponderada: +50)}: La cultura organizacional orientada a la innovación, combinada con programas robustos de capacitación y desarrollo, genera una ventaja competitiva en retención y desarrollo de talento tecnológico especializado.

\item \textbf{Marketing y Ventas (Ponderada: +48)}: El liderazgo en participación de mercado y la fortaleza de la marca MercadoLibre representan barreras de entrada significativas para competidores potenciales.

\item \textbf{Relación con Clientes (Ponderada: +45)}: La personalización de la experiencia del usuario y los sistemas avanzados de gestión de feedback contribuyen a tasas superiores de retención de clientes.
\end{enumerate}

\paragraph{Áreas de Mejora Identificadas}

\begin{enumerate}
\item \textbf{Gestión de Riesgos Financieros (Ponderada: $-4$)}: La exposición a volatilidad cambiaria y riesgos crediticios en operaciones de Mercado Pago requiere fortalecimiento en modelos de gestión de riesgos.

\item \textbf{Optimización de Costos Operativos (Ponderada: 0)}: Existe oportunidad de mejora en la eficiencia de costos operativos, particularmente en logística y servicios de entrega.

\item \textbf{Gestión de Proveedores (Ponderada: +15)}: Aunque funcional, la gestión de la cadena de suministro y relaciones con vendedores presenta oportunidades de optimización y mayor integración estratégica.
\end{enumerate}

\paragraph{Síntesis del Análisis PCI}

El Perfil de Capacidad Interna revela que MercadoLibre posee fortalezas distintivas en las áreas críticas para su modelo de negocio: tecnología, innovación, recursos humanos y marketing. Estas fortalezas están alineadas con las demandas del entorno competitivo y representan capacidades difíciles de replicar.

Las áreas de mejora identificadas no comprometen la posición competitiva actual pero requieren atención estratégica para sostener el crecimiento futuro. La gestión de riesgos financieros y la optimización de costos operativos deben ser prioridades en la agenda estratégica \autocite{barney1991, teece2007, grant2016}.
