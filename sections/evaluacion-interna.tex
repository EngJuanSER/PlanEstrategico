\section{Evaluación Interna -- Perfil de Capacidad Interna (PCI)}
\label{sec:evaluacion_interna}

\subsection{Marco Metodológico PCI}

El Perfil de Capacidad Interna (PCI) es una herramienta de diagnóstico estratégico que permite evaluar sistemáticamente las fortalezas y debilidades organizacionales a través del análisis de capacidades, recursos y competencias internas. Este análisis facilita la identificación de ventajas competitivas sostenibles y áreas de mejora críticas para el desempeño organizacional \autocite{barney1991}.

La metodología PCI se fundamenta en la teoría de recursos y capacidades, que sostiene que las ventajas competitivas duraderas provienen de recursos internos valiosos, raros, difíciles de imitar y debidamente organizados. Para MercadoLibre, este análisis es crucial dado su modelo de negocio que integra tecnología, servicios financieros y operaciones logísticas a escala latinoamericana \autocite{teece2007}.

\subsection{Unidades de Análisis PCI}

Para la evaluación interna de MercadoLibre se han identificado ocho unidades de análisis que representan las capacidades organizacionales críticas:

\begin{enumerate}
\item \textbf{Recursos Humanos}: Gestión del talento, capacitación continua, tasas de retención, clima organizacional y cultura corporativa.

\item \textbf{Tecnología y Sistemas}: Infraestructura tecnológica, automatización de procesos, seguridad informática, escalabilidad de plataformas y capacidades de integración sistémica.

\item \textbf{Capacidades Financieras}: Solidez financiera, liquidez operativa, rentabilidad, acceso a mercados de capital y gestión de flujos de efectivo.

\item \textbf{Procesos Operativos}: Eficiencia operacional, calidad del servicio, gestión de riesgos, optimización logística y tiempos de respuesta.

\item \textbf{Marketing y Ventas}: Participación de mercado, efectividad de canales digitales, satisfacción del cliente y posicionamiento de marca.

\item \textbf{Gestión de Proveedores}: Calidad de proveedores, relaciones estratégicas, gestión de costos y diversificación de la cadena de suministro.

\item \textbf{Innovación y Desarrollo}: Capacidad de innovación, inversión en I+D, desarrollo de nuevos productos y servicios, y velocidad de implementación.

\item \textbf{Relación con Clientes}: Programas de fidelización, calidad del soporte técnico, gestión de feedback y experiencia del usuario.
\end{enumerate}

\subsection{Variables por Unidad de Análisis}

\subsubsection{Recursos Humanos}
\begin{itemize}
\item Programas de capacitación técnica y desarrollo profesional
\item Tasas de retención de talento clave
\item Clima organizacional y satisfacción laboral
\item Cultura de innovación y colaboración
\end{itemize}

\subsubsection{Tecnología y Sistemas}
\begin{itemize}
\item Nivel de automatización de procesos críticos
\item Sistemas de seguridad informática y prevención de fraude
\item Escalabilidad de infraestructura tecnológica
\item Capacidades de integración con sistemas externos
\end{itemize}

\subsubsection{Capacidades Financieras}
\begin{itemize}
\item Índices de liquidez y solvencia
\item Márgenes de rentabilidad (EBITDA, ROE, ROA)
\item Acceso a financiamiento y mercados de capital
\item Gestión de riesgos financieros y cambiarios
\end{itemize}

\subsubsection{Procesos Operativos}
\begin{itemize}
\item Eficiencia en procesamiento de transacciones
\item Calidad del servicio logístico (tiempos de entrega)
\item Sistemas de gestión de riesgos operacionales
\item Optimización de costos operativos
\end{itemize}

\subsubsection{Marketing y Ventas}
\begin{itemize}
\item Participación de mercado en segmentos clave
\item Net Promoter Score (NPS) y satisfacción del cliente
\item Efectividad de marketing digital y adquisición de usuarios
\item Fortaleza de marca y reconocimiento regional
\end{itemize}

\subsubsection{Gestión de Proveedores}
\begin{itemize}
\item Evaluación de calidad de proveedores estratégicos
\item Relaciones de largo plazo con vendedores
\item Gestión de costos y eficiencia de negociación
\item Diversificación de la base de proveedores
\end{itemize}

\subsubsection{Innovación y Desarrollo}
\begin{itemize}
\item Capacidad de desarrollo de nuevos productos
\item Inversión en investigación y desarrollo
\item Proyectos de innovación tecnológica en curso
\item Velocidad de time-to-market para nuevas funcionalidades
\end{itemize}

\subsubsection{Relación con Clientes}
\begin{itemize}
\item Programas de fidelización y retención
\item Calidad del soporte técnico y atención al cliente
\item Sistemas de gestión de feedback y mejora continua
\item Personalización de la experiencia del usuario
\end{itemize}

\subsection{Metodología de Evaluación}

\subsection{Metodología de Evaluación PCI}

La evaluación se realiza mediante la asignación de valores de importancia (1--5) e impacto interno ($-3$ a $+3$) para cada variable:

\begin{itemize}
\item \textbf{Escala de Importancia}: 1 = Muy baja, 2 = Baja, 3 = Media, 4 = Alta, 5 = Muy alta
\item \textbf{Escala de Impacto}: $-3$ = Debilidad crítica, $-2$ = Debilidad significativa, $-1$ = Debilidad menor, 0 = Factor neutro, $+1$ = Fortaleza menor, $+2$ = Fortaleza significativa, $+3$ = Fortaleza distintiva
\end{itemize}

La evaluación ponderada se calcula multiplicando la importancia por el impacto para cada variable analizada.

\subsection{Matriz PCI -- Perfil de Capacidad Interna}

\footnotesize
\begin{longtable}{|p{2.2cm}|p{2.8cm}|c|c|c|c|c|c|c|c|c|c|}
\hline
\multirow{2}{*}{\textbf{Unidad}} & \multirow{2}{*}{\textbf{Variables}} & \multirow{2}{*}{\textbf{Imp.}} & \multirow{2}{*}{\textbf{Impacto}} & \multirow{2}{*}{\textbf{Pond.}} & \multicolumn{7}{c|}{\textbf{Escala de Evaluación}} \\
\cline{6-12}
& & & & & \textbf{-15} & \textbf{-10} & \textbf{-5} & \textbf{0} & \textbf{5} & \textbf{10} & \textbf{15} \\
\hline
\endfirsthead

\multicolumn{12}{c}%
{{\bfseries \tablename\ \thetable{} -- continuación de la página anterior}} \\
\hline
\multirow{2}{*}{\textbf{Unidad}} & \multirow{2}{*}{\textbf{Variables}} & \multirow{2}{*}{\textbf{Imp.}} & \multirow{2}{*}{\textbf{Impacto}} & \multirow{2}{*}{\textbf{Pond.}} & \multicolumn{7}{c|}{\textbf{Escala de Evaluación}} \\
\cline{6-12}
& & & & & \textbf{-15} & \textbf{-10} & \textbf{-5} & \textbf{0} & \textbf{5} & \textbf{10} & \textbf{15} \\
\hline
\endhead

\hline \multicolumn{12}{|r|}{{Continúa en la siguiente página}} \\ \hline
\endfoot

\hline
\endlastfoot
\multirow{4}{*}{\makecell{Recursos\\Humanos}} 
& Capacitación y Desarrollo & 5 & 3 & 15 & \multicolumn{1}{c|}{} & \multicolumn{1}{c|}{} & \multicolumn{1}{c|}{} & \multicolumn{1}{c|}{} & \multicolumn{1}{c|}{} & \multicolumn{1}{c|}{} & o \\
\cline{2-12}
& Retención de Talento & 4 & 2 & 8 & \multicolumn{1}{c|}{} & \multicolumn{1}{c|}{} & \multicolumn{1}{c|}{} & \multicolumn{1}{c|}{} & \multicolumn{1}{c|}{} & \multicolumn{1}{c|}{o} & \\
\cline{2-12}
& Clima Organizacional & 4 & 3 & 12 & \multicolumn{1}{c|}{} & \multicolumn{1}{c|}{} & \multicolumn{1}{c|}{} & \multicolumn{1}{c|}{} & \multicolumn{1}{c|}{} & \multicolumn{1}{c|}{} & o \\
\cline{2-12}
& Cultura de Innovación & 5 & 3 & 15 & \multicolumn{1}{c|}{} & \multicolumn{1}{c|}{} & \multicolumn{1}{c|}{} & \multicolumn{1}{c|}{} & \multicolumn{1}{c|}{} & \multicolumn{1}{c|}{} & o \\
\hline
\multirow{4}{*}{\makecell{Tecnología\\y Sistemas}} 
& Automatización & 5 & 3 & 15 & \multicolumn{1}{c|}{} & \multicolumn{1}{c|}{} & \multicolumn{1}{c|}{} & \multicolumn{1}{c|}{} & \multicolumn{1}{c|}{} & \multicolumn{1}{c|}{} & o \\
\cline{2-12}
& Seguridad Informática & 5 & 3 & 15 & \multicolumn{1}{c|}{} & \multicolumn{1}{c|}{} & \multicolumn{1}{c|}{} & \multicolumn{1}{c|}{} & \multicolumn{1}{c|}{} & \multicolumn{1}{c|}{} & o \\
\cline{2-12}
& Escalabilidad & 5 & 3 & 15 & \multicolumn{1}{c|}{} & \multicolumn{1}{c|}{} & \multicolumn{1}{c|}{} & \multicolumn{1}{c|}{} & \multicolumn{1}{c|}{} & \multicolumn{1}{c|}{} & o \\
\cline{2-12}
& Integración Sistémica & 4 & 2 & 8 & \multicolumn{1}{c|}{} & \multicolumn{1}{c|}{} & \multicolumn{1}{c|}{} & \multicolumn{1}{c|}{} & \multicolumn{1}{c|}{} & \multicolumn{1}{c|}{o} & \\
\hline
\multirow{4}{*}{\makecell{Capacidades\\Financieras}} 
& Liquidez y Solvencia & 5 & 2 & 10 & \multicolumn{1}{c|}{} & \multicolumn{1}{c|}{} & \multicolumn{1}{c|}{} & \multicolumn{1}{c|}{} & \multicolumn{1}{c|}{} & \multicolumn{1}{c|}{o} & \\
\cline{2-12}
& Rentabilidad & 4 & 1 & 4 & \multicolumn{1}{c|}{} & \multicolumn{1}{c|}{} & \multicolumn{1}{c|}{} & \multicolumn{1}{c|}{} & \multicolumn{1}{c|}{o} & \multicolumn{1}{c|}{} & \\
\cline{2-12}
& Acceso a Financiamiento & 5 & 3 & 15 & \multicolumn{1}{c|}{} & \multicolumn{1}{c|}{} & \multicolumn{1}{c|}{} & \multicolumn{1}{c|}{} & \multicolumn{1}{c|}{} & \multicolumn{1}{c|}{} & o \\
\cline{2-12}
& Gestión de Riesgos & 4 & -1 & -4 & \multicolumn{1}{c|}{} & \multicolumn{1}{c|}{} & \multicolumn{1}{c|}{o} & \multicolumn{1}{c|}{} & \multicolumn{1}{c|}{} & \multicolumn{1}{c|}{} & \\
\hline
\multirow{4}{*}{\makecell{Procesos\\Operativos}} 
& Eficiencia Operacional & 5 & 2 & 10 & \multicolumn{1}{c|}{} & \multicolumn{1}{c|}{} & \multicolumn{1}{c|}{} & \multicolumn{1}{c|}{} & \multicolumn{1}{c|}{} & \multicolumn{1}{c|}{o} & \\
\cline{2-12}
& Calidad del Servicio & 5 & 3 & 15 & \multicolumn{1}{c|}{} & \multicolumn{1}{c|}{} & \multicolumn{1}{c|}{} & \multicolumn{1}{c|}{} & \multicolumn{1}{c|}{} & \multicolumn{1}{c|}{} & o \\
\cline{2-12}
& Gestión de Riesgos Op. & 4 & 1 & 4 & \multicolumn{1}{c|}{} & \multicolumn{1}{c|}{} & \multicolumn{1}{c|}{} & \multicolumn{1}{c|}{} & \multicolumn{1}{c|}{o} & \multicolumn{1}{c|}{} & \\
\cline{2-12}
& Optimización Costos & 3 & 0 & 0 & \multicolumn{1}{c|}{} & \multicolumn{1}{c|}{} & \multicolumn{1}{c|}{} & \multicolumn{1}{c|}{o} & \multicolumn{1}{c|}{} & \multicolumn{1}{c|}{} & \\
\hline
\multirow{4}{*}{\makecell{Marketing\\y Ventas}} 
& Participación de Mercado & 5 & 3 & 15 & \multicolumn{1}{c|}{} & \multicolumn{1}{c|}{} & \multicolumn{1}{c|}{} & \multicolumn{1}{c|}{} & \multicolumn{1}{c|}{} & \multicolumn{1}{c|}{} & o \\
\cline{2-12}
& Satisfacción del Cliente & 5 & 2 & 10 & \multicolumn{1}{c|}{} & \multicolumn{1}{c|}{} & \multicolumn{1}{c|}{} & \multicolumn{1}{c|}{} & \multicolumn{1}{c|}{} & \multicolumn{1}{c|}{o} & \\
\cline{2-12}
& Efectividad Digital & 4 & 2 & 8 & \multicolumn{1}{c|}{} & \multicolumn{1}{c|}{} & \multicolumn{1}{c|}{} & \multicolumn{1}{c|}{} & \multicolumn{1}{c|}{} & \multicolumn{1}{c|}{o} & \\
\cline{2-12}
& Fortaleza de Marca & 5 & 3 & 15 & \multicolumn{1}{c|}{} & \multicolumn{1}{c|}{} & \multicolumn{1}{c|}{} & \multicolumn{1}{c|}{} & \multicolumn{1}{c|}{} & \multicolumn{1}{c|}{} & o \\
\hline
\multirow{3}{*}{\makecell{Gestión de\\Proveedores}} 
& Calidad Proveedores & 4 & 2 & 8 & \multicolumn{1}{c|}{} & \multicolumn{1}{c|}{} & \multicolumn{1}{c|}{} & \multicolumn{1}{c|}{} & \multicolumn{1}{c|}{} & \multicolumn{1}{c|}{o} & \\
\cline{2-12}
& Relaciones Estratégicas & 3 & 1 & 3 & \multicolumn{1}{c|}{} & \multicolumn{1}{c|}{} & \multicolumn{1}{c|}{} & \multicolumn{1}{c|}{} & \multicolumn{1}{c|}{o} & \multicolumn{1}{c|}{} & \\
\cline{2-12}
& Gestión de Costos & 4 & 1 & 4 & \multicolumn{1}{c|}{} & \multicolumn{1}{c|}{} & \multicolumn{1}{c|}{} & \multicolumn{1}{c|}{} & \multicolumn{1}{c|}{o} & \multicolumn{1}{c|}{} & \\
\hline
\multirow{4}{*}{\makecell{Innovación\\y Desarrollo}} 
& Capacidad de Innovación & 5 & 3 & 15 & \multicolumn{1}{c|}{} & \multicolumn{1}{c|}{} & \multicolumn{1}{c|}{} & \multicolumn{1}{c|}{} & \multicolumn{1}{c|}{} & \multicolumn{1}{c|}{} & o \\
\cline{2-12}
& Inversión en I+D & 5 & 3 & 15 & \multicolumn{1}{c|}{} & \multicolumn{1}{c|}{} & \multicolumn{1}{c|}{} & \multicolumn{1}{c|}{} & \multicolumn{1}{c|}{} & \multicolumn{1}{c|}{} & o \\
\cline{2-12}
& Proyectos Innovación & 4 & 3 & 12 & \multicolumn{1}{c|}{} & \multicolumn{1}{c|}{} & \multicolumn{1}{c|}{} & \multicolumn{1}{c|}{} & \multicolumn{1}{c|}{} & \multicolumn{1}{c|}{} & o \\
\cline{2-12}
& Velocidad Time-to-Market & 4 & 2 & 8 & \multicolumn{1}{c|}{} & \multicolumn{1}{c|}{} & \multicolumn{1}{c|}{} & \multicolumn{1}{c|}{} & \multicolumn{1}{c|}{} & \multicolumn{1}{c|}{o} & \\
\hline
\multirow{4}{*}{\makecell{Relación\\con Clientes}} 
& Programas Fidelización & 4 & 2 & 8 & \multicolumn{1}{c|}{} & \multicolumn{1}{c|}{} & \multicolumn{1}{c|}{} & \multicolumn{1}{c|}{} & \multicolumn{1}{c|}{} & \multicolumn{1}{c|}{o} & \\
\cline{2-12}
& Calidad de Soporte & 5 & 2 & 10 & \multicolumn{1}{c|}{} & \multicolumn{1}{c|}{} & \multicolumn{1}{c|}{} & \multicolumn{1}{c|}{} & \multicolumn{1}{c|}{} & \multicolumn{1}{c|}{o} & \\
\cline{2-12}
& Gestión de Feedback & 4 & 3 & 12 & \multicolumn{1}{c|}{} & \multicolumn{1}{c|}{} & \multicolumn{1}{c|}{} & \multicolumn{1}{c|}{} & \multicolumn{1}{c|}{} & \multicolumn{1}{c|}{} & o \\
\cline{2-12}
& Personalización UX & 5 & 3 & 15 & \multicolumn{1}{c|}{} & \multicolumn{1}{c|}{} & \multicolumn{1}{c|}{} & \multicolumn{1}{c|}{} & \multicolumn{1}{c|}{} & \multicolumn{1}{c|}{} & o \\
\hline
\caption{Matriz PCI -- Perfil de Capacidad Interna MercadoLibre}
\label{tab:matriz_pci}
\end{longtable}

La Tabla \ref{tab:matriz_pci} presenta la evaluación detallada de todas las variables consideradas en el análisis PCI, permitiendo identificar las fortalezas y debilidades organizacionales de MercadoLibre.

\subsection{Resultados Consolidados por Unidad de Análisis}

\begin{table}[H]
\centering
\begin{tabular}{|l|c|c|}
\hline
\textbf{Unidad de Análisis} & \textbf{Ponderación Total} & \textbf{Clasificación} \\
\hline
Innovación y Desarrollo & +50 & Fortaleza Distintiva \\
\hline
Tecnología y Sistemas & +53 & Fortaleza Distintiva \\
\hline
Marketing y Ventas & +48 & Fortaleza Distintiva \\
\hline
Recursos Humanos & +50 & Fortaleza Distintiva \\
\hline
Relación con Clientes & +45 & Fortaleza Significativa \\
\hline
Procesos Operativos & +29 & Fortaleza Significativa \\
\hline
Capacidades Financieras & +25 & Fortaleza Significativa \\
\hline
Gestión de Proveedores & +15 & Fortaleza Minor \\
\hline
\end{tabular}
\caption{Evaluación Ponderada por Unidad de Análisis -- PCI}
\label{tab:resultados_pci}
\end{table}

La Tabla \ref{tab:resultados_pci} consolida los resultados ponderados por cada unidad de análisis, mostrando que MercadoLibre posee fortalezas distintivas en las áreas críticas para su modelo de negocio.

\subsection{Identificación de Fortalezas y Debilidades}

\subsubsection{Principales Fortalezas Identificadas}

\begin{enumerate}
\item \textbf{Tecnología y Sistemas (Ponderada: +53)}: La infraestructura tecnológica de MercadoLibre constituye su ventaja competitiva más robusta, con altos niveles de automatización, seguridad informática avanzada y escalabilidad demostrada para soportar millones de transacciones diarias.

\item \textbf{Innovación y Desarrollo (Ponderada: +50)}: La capacidad continua de innovación y la inversión sostenida en I+D permiten a MercadoLibre mantener liderazgo en desarrollo de nuevos productos y servicios financieros.

\item \textbf{Recursos Humanos (Ponderada: +50)}: La cultura organizacional orientada a la innovación, combinada con programas robustos de capacitación y desarrollo, genera una ventaja competitiva en retención y desarrollo de talento tecnológico especializado.

\item \textbf{Marketing y Ventas (Ponderada: +48)}: El liderazgo en participación de mercado y la fortaleza de la marca MercadoLibre representan barreras de entrada significativas para competidores potenciales.

\item \textbf{Relación con Clientes (Ponderada: +45)}: La personalización de la experiencia del usuario y los sistemas avanzados de gestión de feedback contribuyen a tasas superiores de retención de clientes.
\end{enumerate}

\subsubsection{Áreas de Mejora Identificadas}

\begin{enumerate}
\item \textbf{Gestión de Riesgos Financieros (Ponderada: $-4$)}: La exposición a volatilidad cambiaria y riesgos crediticios en operaciones de Mercado Pago requiere fortalecimiento en modelos de gestión de riesgos.

\item \textbf{Optimización de Costos Operativos (Ponderada: 0)}: Existe oportunidad de mejora en la eficiencia de costos operativos, particularmente en logística y servicios de entrega.

\item \textbf{Gestión de Proveedores (Ponderada: +15)}: Aunque funcional, la gestión de la cadena de suministro y relaciones con vendedores presenta oportunidades de optimización y mayor integración estratégica.
\end{enumerate}

\subsection{Conclusiones del Análisis PCI}

El Perfil de Capacidad Interna revela que MercadoLibre posee fortalezas distintivas en las áreas críticas para su modelo de negocio: tecnología, innovación, recursos humanos y marketing. Estas fortalezas están alineadas con las demandas del entorno competitivo y representan capacidades difíciles de replicar.

Las áreas de mejora identificadas no comprometen la posición competitiva actual pero requieren atención estratégica para sostener el crecimiento futuro. La gestión de riesgos financieros y la optimización de costos operativos deben ser prioridades en la agenda estratégica \autocite{barney1991, teece2007, grant2016}.
