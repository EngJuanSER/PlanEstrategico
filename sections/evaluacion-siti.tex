\section{Evaluación de Sistemas de Información y Tecnología (SI/TI)}
\label{sec:evaluacion_siti}

\subsection{Marco Metodológico de Evaluación SI/TI}

La evaluación de Sistemas de Información y Tecnología de la Información (SI/TI) es un análisis especializado que examina cómo las capacidades tecnológicas y los sistemas de información soportan y habilitan las estrategias organizacionales. Este análisis es particularmente crítico para empresas de base tecnológica como MercadoLibre, donde la tecnología no solo es un habilitador sino el núcleo mismo del modelo de negocio \autocite{porter1985}.

La metodología de evaluación SI/TI se fundamenta en el marco de alineación estratégica que sostiene que el valor de la tecnología se maximiza cuando existe coherencia entre la estrategia de negocio, la estrategia de TI, la infraestructura organizacional y la infraestructura tecnológica. Para MercadoLibre, esta alineación determina su capacidad de escalar operaciones, innovar continuamente y mantener ventajas competitivas sostenibles \autocite{teece2007}.

\subsection{Dimensiones de Análisis SI/TI}

La evaluación SI/TI de MercadoLibre se estructura en ocho dimensiones que cubren tanto aspectos tecnológicos como de gestión de sistemas de información:

\begin{enumerate}
\item \textbf{Recursos Humanos TI}: Estructura organizacional del área de tecnología, formación técnica especializada, tasas de rotación de personal tecnológico y capacidades de gestión de talento digital.

\item \textbf{Tecnología y Sistemas}: Infraestructura tecnológica actual, adopción de tecnologías emergentes, sistemas de mantenimiento preventivo y capacidad de respuesta ante incidentes.

\item \textbf{Capacidades Financieras TI}: Control presupuestal de proyectos tecnológicos, gestión de riesgos financieros en inversiones TI, política de inversión en innovación y retorno sobre inversión tecnológica.

\item \textbf{Procesos Operativos TI}: Documentación de procesos tecnológicos, nivel de automatización, sistemas de monitoreo de KPIs tecnológicos y gestión de incidentes.

\item \textbf{Marketing y Ventas Digital}: Estrategia de marketing digital, segmentación basada en datos, analítica avanzada y sistemas de inteligencia de mercado.

\item \textbf{Gestión de Proveedores Tecnológicos}: Evaluación de desempeño de proveedores de tecnología, gestión de contratos de servicios cloud, identificación de proveedores alternativos y gestión de dependencias tecnológicas.

\item \textbf{Innovación y Desarrollo Tecnológico}: Estrategia de I+D tecnológico, gestión de patentes y propiedad intelectual, colaboraciones con ecosistema de innovación y adopción de metodologías ágiles.

\item \textbf{Relación con Clientes mediante TI}: Sistemas CRM, gestión automatizada de retroalimentación, sistemas de resolución de problemas y personalización mediante machine learning.
\end{enumerate}

\subsection{Variables por Dimensión SI/TI}

\subsubsection{Recursos Humanos TI}
\begin{itemize}
\item Estructura organizacional del área tecnológica
\item Programas de formación técnica especializada
\item Tasas de rotación de personal TI
\item Atracción y retención de talento tecnológico
\end{itemize}

\subsubsection{Tecnología y Sistemas}
\begin{itemize}
\item Infraestructura cloud y capacidad de cómputo
\item Adopción de tecnologías emergentes (IA, ML, blockchain)
\item Sistemas de mantenimiento y monitoreo
\item Capacidad de respuesta ante incidentes críticos
\end{itemize}

\subsubsection{Capacidades Financieras TI}
\begin{itemize}
\item Control presupuestal de proyectos tecnológicos
\item Gestión de riesgos en inversiones TI
\item Política de capitalización de desarrollos
\item ROI de proyectos tecnológicos
\end{itemize}

\subsubsection{Procesos Operativos TI}
\begin{itemize}
\item Documentación de arquitecturas y procesos
\item Nivel de automatización de operaciones TI
\item Sistemas de monitoreo de KPIs tecnológicos
\item Gestión de cambios y releases
\end{itemize}

\subsubsection{Marketing y Ventas Digital}
\begin{itemize}
\item Estrategia de marketing basada en datos
\item Segmentación avanzada de usuarios
\item Sistemas de analítica y business intelligence
\item Personalización de campañas mediante ML
\end{itemize}

\subsubsection{Gestión de Proveedores Tecnológicos}
\begin{itemize}
\item Evaluación de desempeño de proveedores cloud
\item Gestión de contratos SLA
\item Identificación de alternativas tecnológicas
\item Gestión de dependencias y vendor lock-in
\end{itemize}

\subsubsection{Innovación y Desarrollo Tecnológico}
\begin{itemize}
\item Estrategia de innovación tecnológica
\item Gestión de propiedad intelectual tecnológica
\item Colaboraciones con universidades y startups
\item Adopción de metodologías ágiles y DevOps
\end{itemize}

\subsubsection{Relación con Clientes mediante TI}
\begin{itemize}
\item Sistemas CRM y gestión de interacciones
\item Automatización de atención al cliente
\item Sistemas de análisis de feedback
\item Personalización de experiencia mediante ML
\end{itemize}

\subsection{Metodología de Evaluación SI/TI}

La evaluación utiliza la misma escala de importancia e impacto que el análisis PCI, con énfasis en la contribución tecnológica a las ventajas competitivas:

\begin{itemize}
\item \textbf{Escala de Importancia Estratégica}: 1 = Muy baja, 2 = Baja, 3 = Media, 4 = Alta, 5 = Muy alta
\item \textbf{Escala de Madurez y Capacidad}: $-3$ = Brecha crítica, $-2$ = Brecha significativa, $-1$ = Brecha menor, 0 = Capacidad adecuada básica, $+1$ = Capacidad superior, $+2$ = Capacidad avanzada, $+3$ = Capacidad de clase mundial
\end{itemize}

La evaluación ponderada se calcula multiplicando la importancia por la capacidad para cada variable analizada.

\subsection{Matriz SI/TI -- Perfil Tecnológico}

\footnotesize
\begin{longtable}{|p{2.2cm}|p{2.8cm}|c|c|c|c|c|c|c|c|c|c|}
\hline
\multirow{2}{*}{\textbf{Dimensión}} & \multirow{2}{*}{\textbf{Variables}} & \multirow{2}{*}{\textbf{Imp.}} & \multirow{2}{*}{\textbf{Capac.}} & \multirow{2}{*}{\textbf{Pond.}} & \multicolumn{7}{c|}{\textbf{Escala de Evaluación}} \\
\cline{6-12}
& & & & & \textbf{-15} & \textbf{-10} & \textbf{-5} & \textbf{0} & \textbf{5} & \textbf{10} & \textbf{15} \\
\hline
\endfirsthead

\multicolumn{12}{c}%
{{\bfseries \tablename\ \thetable{} -- continuación de la página anterior}} \\
\hline
\multirow{2}{*}{\textbf{Dimensión}} & \multirow{2}{*}{\textbf{Variables}} & \multirow{2}{*}{\textbf{Imp.}} & \multirow{2}{*}{\textbf{Capac.}} & \multirow{2}{*}{\textbf{Pond.}} & \multicolumn{7}{c|}{\textbf{Escala de Evaluación}} \\
\cline{6-12}
& & & & & \textbf{-15} & \textbf{-10} & \textbf{-5} & \textbf{0} & \textbf{5} & \textbf{10} & \textbf{15} \\
\hline
\endhead

\hline \multicolumn{12}{|r|}{{Continúa en la siguiente página}} \\ \hline
\endfoot

\hline
\endlastfoot
\multirow{4}{*}{\makecell{Recursos\\Humanos TI}} 
& Estructura Organizacional & 4 & 2 & 8 & \multicolumn{1}{c|}{} & \multicolumn{1}{c|}{} & \multicolumn{1}{c|}{} & \multicolumn{1}{c|}{} & \multicolumn{1}{c|}{} & \multicolumn{1}{c|}{o} & \\
\cline{2-12}
& Formación Técnica & 5 & 3 & 15 & \multicolumn{1}{c|}{} & \multicolumn{1}{c|}{} & \multicolumn{1}{c|}{} & \multicolumn{1}{c|}{} & \multicolumn{1}{c|}{} & \multicolumn{1}{c|}{} & o \\
\cline{2-12}
& Rotación Personal TI & 4 & -1 & -4 & \multicolumn{1}{c|}{} & \multicolumn{1}{c|}{} & \multicolumn{1}{c|}{o} & \multicolumn{1}{c|}{} & \multicolumn{1}{c|}{} & \multicolumn{1}{c|}{} & \\
\cline{2-12}
& Atracción Talento & 5 & 2 & 10 & \multicolumn{1}{c|}{} & \multicolumn{1}{c|}{} & \multicolumn{1}{c|}{} & \multicolumn{1}{c|}{} & \multicolumn{1}{c|}{} & \multicolumn{1}{c|}{o} & \\
\hline
\multirow{4}{*}{\makecell{Tecnología\\y Sistemas}} 
& Infraestructura Cloud & 5 & 3 & 15 & \multicolumn{1}{c|}{} & \multicolumn{1}{c|}{} & \multicolumn{1}{c|}{} & \multicolumn{1}{c|}{} & \multicolumn{1}{c|}{} & \multicolumn{1}{c|}{} & o \\
\cline{2-12}
& Adopción Tec. Emergentes & 5 & 3 & 15 & \multicolumn{1}{c|}{} & \multicolumn{1}{c|}{} & \multicolumn{1}{c|}{} & \multicolumn{1}{c|}{} & \multicolumn{1}{c|}{} & \multicolumn{1}{c|}{} & o \\
\cline{2-12}
& Sistemas Mantenimiento & 4 & 2 & 8 & \multicolumn{1}{c|}{} & \multicolumn{1}{c|}{} & \multicolumn{1}{c|}{} & \multicolumn{1}{c|}{} & \multicolumn{1}{c|}{} & \multicolumn{1}{c|}{o} & \\
\cline{2-12}
& Respuesta a Incidentes & 5 & 3 & 15 & \multicolumn{1}{c|}{} & \multicolumn{1}{c|}{} & \multicolumn{1}{c|}{} & \multicolumn{1}{c|}{} & \multicolumn{1}{c|}{} & \multicolumn{1}{c|}{} & o \\
\hline
\multirow{4}{*}{\makecell{Capacidades\\Financieras TI}} 
& Control Presupuestal & 4 & 1 & 4 & \multicolumn{1}{c|}{} & \multicolumn{1}{c|}{} & \multicolumn{1}{c|}{} & \multicolumn{1}{c|}{} & \multicolumn{1}{c|}{o} & \multicolumn{1}{c|}{} & \\
\cline{2-12}
& Gestión Riesgos TI & 4 & 0 & 0 & \multicolumn{1}{c|}{} & \multicolumn{1}{c|}{} & \multicolumn{1}{c|}{} & \multicolumn{1}{c|}{o} & \multicolumn{1}{c|}{} & \multicolumn{1}{c|}{} & \\
\cline{2-12}
& Política de Inversión & 5 & 2 & 10 & \multicolumn{1}{c|}{} & \multicolumn{1}{c|}{} & \multicolumn{1}{c|}{} & \multicolumn{1}{c|}{} & \multicolumn{1}{c|}{} & \multicolumn{1}{c|}{o} & \\
\cline{2-12}
& ROI Proyectos TI & 3 & -1 & -3 & \multicolumn{1}{c|}{} & \multicolumn{1}{c|}{} & \multicolumn{1}{c|}{o} & \multicolumn{1}{c|}{} & \multicolumn{1}{c|}{} & \multicolumn{1}{c|}{} & \\
\hline
\multirow{4}{*}{\makecell{Procesos\\Operativos TI}} 
& Documentación & 4 & 2 & 8 & \multicolumn{1}{c|}{} & \multicolumn{1}{c|}{} & \multicolumn{1}{c|}{} & \multicolumn{1}{c|}{} & \multicolumn{1}{c|}{} & \multicolumn{1}{c|}{o} & \\
\cline{2-12}
& Automatización TI & 5 & 3 & 15 & \multicolumn{1}{c|}{} & \multicolumn{1}{c|}{} & \multicolumn{1}{c|}{} & \multicolumn{1}{c|}{} & \multicolumn{1}{c|}{} & \multicolumn{1}{c|}{} & o \\
\cline{2-12}
& Monitoreo KPIs & 5 & 3 & 15 & \multicolumn{1}{c|}{} & \multicolumn{1}{c|}{} & \multicolumn{1}{c|}{} & \multicolumn{1}{c|}{} & \multicolumn{1}{c|}{} & \multicolumn{1}{c|}{} & o \\
\cline{2-12}
& Gestión de Cambios & 4 & 2 & 8 & \multicolumn{1}{c|}{} & \multicolumn{1}{c|}{} & \multicolumn{1}{c|}{} & \multicolumn{1}{c|}{} & \multicolumn{1}{c|}{} & \multicolumn{1}{c|}{o} & \\
\hline
\multirow{4}{*}{\makecell{Marketing\\Digital}} 
& Estrategia Data-Driven & 5 & 3 & 15 & \multicolumn{1}{c|}{} & \multicolumn{1}{c|}{} & \multicolumn{1}{c|}{} & \multicolumn{1}{c|}{} & \multicolumn{1}{c|}{} & \multicolumn{1}{c|}{} & o \\
\cline{2-12}
& Segmentación Avanzada & 5 & 3 & 15 & \multicolumn{1}{c|}{} & \multicolumn{1}{c|}{} & \multicolumn{1}{c|}{} & \multicolumn{1}{c|}{} & \multicolumn{1}{c|}{} & \multicolumn{1}{c|}{} & o \\
\cline{2-12}
& Analítica y BI & 5 & 3 & 15 & \multicolumn{1}{c|}{} & \multicolumn{1}{c|}{} & \multicolumn{1}{c|}{} & \multicolumn{1}{c|}{} & \multicolumn{1}{c|}{} & \multicolumn{1}{c|}{} & o \\
\cline{2-12}
& Personalización ML & 5 & 2 & 10 & \multicolumn{1}{c|}{} & \multicolumn{1}{c|}{} & \multicolumn{1}{c|}{} & \multicolumn{1}{c|}{} & \multicolumn{1}{c|}{} & \multicolumn{1}{c|}{o} & \\
\hline
\multirow{4}{*}{\makecell{Gestión\\Proveed. TI}} 
& Evaluación Proveedores & 4 & 2 & 8 & \multicolumn{1}{c|}{} & \multicolumn{1}{c|}{} & \multicolumn{1}{c|}{} & \multicolumn{1}{c|}{} & \multicolumn{1}{c|}{} & \multicolumn{1}{c|}{o} & \\
\cline{2-12}
& Gestión Contratos SLA & 4 & 1 & 4 & \multicolumn{1}{c|}{} & \multicolumn{1}{c|}{} & \multicolumn{1}{c|}{} & \multicolumn{1}{c|}{} & \multicolumn{1}{c|}{o} & \multicolumn{1}{c|}{} & \\
\cline{2-12}
& Alternativas Tecnológicas & 3 & 1 & 3 & \multicolumn{1}{c|}{} & \multicolumn{1}{c|}{} & \multicolumn{1}{c|}{} & \multicolumn{1}{c|}{} & \multicolumn{1}{c|}{o} & \multicolumn{1}{c|}{} & \\
\cline{2-12}
& Gestión Dependencias & 4 & 0 & 0 & \multicolumn{1}{c|}{} & \multicolumn{1}{c|}{} & \multicolumn{1}{c|}{} & \multicolumn{1}{c|}{o} & \multicolumn{1}{c|}{} & \multicolumn{1}{c|}{} & \\
\hline
\multirow{4}{*}{\makecell{Innovación\\y Des. Tec.}} 
& Estrategia I+D Tecnológica & 5 & 3 & 15 & \multicolumn{1}{c|}{} & \multicolumn{1}{c|}{} & \multicolumn{1}{c|}{} & \multicolumn{1}{c|}{} & \multicolumn{1}{c|}{} & \multicolumn{1}{c|}{} & o \\
\cline{2-12}
& Gestión Propiedad Intelectual & 4 & 2 & 8 & \multicolumn{1}{c|}{} & \multicolumn{1}{c|}{} & \multicolumn{1}{c|}{} & \multicolumn{1}{c|}{} & \multicolumn{1}{c|}{} & \multicolumn{1}{c|}{o} & \\
\cline{2-12}
& Colaboraciones Innovación & 5 & 3 & 15 & \multicolumn{1}{c|}{} & \multicolumn{1}{c|}{} & \multicolumn{1}{c|}{} & \multicolumn{1}{c|}{} & \multicolumn{1}{c|}{} & \multicolumn{1}{c|}{} & o \\
\cline{2-12}
& Metodologías Ágiles & 5 & 3 & 15 & \multicolumn{1}{c|}{} & \multicolumn{1}{c|}{} & \multicolumn{1}{c|}{} & \multicolumn{1}{c|}{} & \multicolumn{1}{c|}{} & \multicolumn{1}{c|}{} & o \\
\hline
\multirow{4}{*}{\makecell{Relación\\Clientes TI}} 
& Sistemas CRM & 5 & 3 & 15 & \multicolumn{1}{c|}{} & \multicolumn{1}{c|}{} & \multicolumn{1}{c|}{} & \multicolumn{1}{c|}{} & \multicolumn{1}{c|}{} & \multicolumn{1}{c|}{} & o \\
\cline{2-12}
& Automatización Atención & 5 & 3 & 15 & \multicolumn{1}{c|}{} & \multicolumn{1}{c|}{} & \multicolumn{1}{c|}{} & \multicolumn{1}{c|}{} & \multicolumn{1}{c|}{} & \multicolumn{1}{c|}{} & o \\
\cline{2-12}
& Análisis de Feedback & 4 & 3 & 12 & \multicolumn{1}{c|}{} & \multicolumn{1}{c|}{} & \multicolumn{1}{c|}{} & \multicolumn{1}{c|}{} & \multicolumn{1}{c|}{} & \multicolumn{1}{c|}{} & o \\
\cline{2-12}
& Personalización ML & 5 & 3 & 15 & \multicolumn{1}{c|}{} & \multicolumn{1}{c|}{} & \multicolumn{1}{c|}{} & \multicolumn{1}{c|}{} & \multicolumn{1}{c|}{} & \multicolumn{1}{c|}{} & o \\
\hline
\caption{Matriz SI/TI -- Perfil de Sistemas de Información y Tecnología MercadoLibre}
\label{tab:matriz_siti}
\end{longtable}

La Tabla \ref{tab:matriz_siti} presenta la evaluación detallada de las capacidades tecnológicas de MercadoLibre, considerando tanto la infraestructura como los sistemas de información que soportan su modelo de negocio.

\subsection{Resultados Consolidados por Dimensión SI/TI}

\begin{table}[H]
\centering
\begin{tabular}{|l|c|c|}
\hline
\textbf{Dimensión SI/TI} & \textbf{Ponderación Total} & \textbf{Clasificación} \\
\hline
Relación con Clientes mediante TI & +57 & Capacidad de Clase Mundial \\
\hline
Marketing Digital & +55 & Capacidad de Clase Mundial \\
\hline
Innovación y Desarrollo Tecnológico & +53 & Capacidad de Clase Mundial \\
\hline
Tecnología y Sistemas & +53 & Capacidad de Clase Mundial \\
\hline
Procesos Operativos TI & +46 & Capacidad Avanzada \\
\hline
Recursos Humanos TI & +29 & Capacidad Avanzada \\
\hline
Gestión de Proveedores TI & +15 & Capacidad Adecuada \\
\hline
Capacidades Financieras TI & +11 & Capacidad Adecuada \\
\hline
\end{tabular}
\caption{Evaluación Ponderada por Dimensión SI/TI}
\label{tab:resultados_siti}
\end{table}

La Tabla \ref{tab:resultados_siti} consolida las evaluaciones por dimensión tecnológica, evidenciando que MercadoLibre posee capacidades de clase mundial en las áreas críticas de su estrategia digital.

\subsection{Análisis Comparativo PCI vs SI/TI}

\begin{table}[H]
\centering
\small
\begin{tabular}{|l|c|c|c|}
\hline
\textbf{Unidad de Análisis} & \textbf{PCI Ponderada} & \textbf{SI/TI Ponderada} & \textbf{Diferencial} \\
\hline
Recursos Humanos / RH TI & +50 & +29 & -21 \\
\hline
Tecnología y Sistemas & +53 & +53 & 0 \\
\hline
Capacidades Financieras / Fin. TI & +25 & +11 & -14 \\
\hline
Procesos Operativos / Proc. TI & +29 & +46 & +17 \\
\hline
Marketing y Ventas / Mkt. Digital & +48 & +55 & +7 \\
\hline
Gestión de Proveedores / Prov. TI & +15 & +15 & 0 \\
\hline
Innovación y Desarrollo / Inn. Tec. & +50 & +53 & +3 \\
\hline
Relación con Clientes / RC mediante TI & +45 & +57 & +12 \\
\hline
\end{tabular}
\caption{Comparación de Evaluaciones PCI y SI/TI por Unidad de Análisis}
\label{tab:comparacion_pci_siti}
\end{table}

La Tabla \ref{tab:comparacion_pci_siti} permite identificar las convergencias y divergencias entre las capacidades organizacionales generales y las capacidades tecnológicas específicas de MercadoLibre.

\subsection{Principales Hallazgos de la Evaluación SI/TI}

\subsubsection{Fortalezas Tecnológicas Distintivas}

\begin{enumerate}
\item \textbf{Relación con Clientes mediante TI (Ponderada: +57)}: Los sistemas CRM, la automatización de atención y la personalización mediante machine learning posicionan a MercadoLibre como líder regional en experiencia digital del cliente.

\item \textbf{Marketing Digital (Ponderada: +55)}: Las capacidades de analítica avanzada, segmentación basada en datos y personalización de campañas representan ventajas competitivas sostenibles.

\item \textbf{Innovación y Desarrollo Tecnológico (Ponderada: +53)}: La estrategia de I+D tecnológica, combinada con colaboraciones con el ecosistema de innovación y adopción de metodologías ágiles, garantiza capacidad de innovación continua.

\item \textbf{Tecnología y Sistemas (Ponderada: +53)}: La infraestructura cloud de clase mundial y la adopción temprana de tecnologías emergentes constituyen barreras de entrada significativas.
\end{enumerate}

\subsubsection{Áreas de Desarrollo Tecnológico}

\begin{enumerate}
\item \textbf{ROI de Proyectos TI (Ponderada: $-3$)}: Existe oportunidad de mejorar los mecanismos de medición y evaluación del retorno sobre inversión en proyectos tecnológicos.

\item \textbf{Rotación de Personal TI (Ponderada: $-4$)}: La competencia por talento tecnológico genera presiones en retención que requieren estrategias específicas de gestión de recursos humanos.

\item \textbf{Gestión de Dependencias Tecnológicas (Ponderada: 0)}: La dependencia de proveedores cloud específicos requiere estrategias de mitigación de riesgos de vendor lock-in.
\end{enumerate}

\subsection{Insights del Análisis Comparativo}

El análisis comparativo entre PCI y SI/TI revela patrones importantes:

\begin{itemize}
\item \textbf{Convergencia en Tecnología}: La evaluación idéntica (+53) en ambos análisis confirma que la tecnología es el núcleo competitivo de MercadoLibre.

\item \textbf{Superioridad SI/TI en Procesos}: La mayor evaluación en procesos operativos TI (+46 vs +29) indica que la automatización y digitalización han avanzado más que la optimización de procesos tradicionales.

\item \textbf{Superioridad SI/TI en Relación con Clientes}: El diferencial de +12 puntos indica que las capacidades tecnológicas de CRM y personalización superan las capacidades tradicionales de gestión de clientes.

\item \textbf{Brecha en Recursos Humanos}: El diferencial negativo de $-21$ puntos sugiere que las capacidades de gestión de talento tecnológico requieren atención estratégica.
\end{itemize}

\subsection{Conclusiones de la Evaluación SI/TI}

La evaluación SI/TI confirma que MercadoLibre posee capacidades tecnológicas de clase mundial en las dimensiones críticas para su modelo de negocio. La tecnología no solo es un habilitador sino el diferenciador competitivo principal.

Las capacidades superiores en marketing digital, relación con clientes mediante TI e innovación tecnológica se alinean perfectamente con las demandas del mercado latinoamericano de comercio electrónico y servicios financieros digitales.

Las áreas de desarrollo identificadas (ROI de proyectos TI, retención de talento tecnológico y gestión de dependencias) son manejables y no comprometen la posición competitiva actual. Sin embargo, requieren atención en el horizonte de planificación estratégica de mediano plazo \autocite{porter1985, teece2007, grant2016}.
